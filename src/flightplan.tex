%!TEX root = index.tex

\section{Flight Plan}

Every controlled aircraft has its flight plan (FP) that contains basic information about the flight. These information include aircraft identification, aircraft type, departure point, departure time, route of flight, destination, fuel on board and a few other. \cite[Chapter 5]{aim} The purpose of the flight plan is to give all relevant data about the aircraft to the air traffic controller in a standardized format so it can be used efficiently during the flight control. The flight plan is handed over from one control to another as the aircraft progresses from its departure airport to destination along the flight route.

The flight route is a sequence of fixes and routes the plane will fly through during its flight.

\subsection{Fix}
Fix is named point on Earth that aircraft use for their navigation. The position of fix can be determined by radio beacons or as GPS position for airplanes equipped with GPS receivers.

If aircraft has certain fix in its flight plan, it must fly over the position of the fix during the flight. This way the route of the aircraft is defined and automatic tools can be used to notify the pilot and air traffic controller if the plane deflects too far from the trajectory defined by its flight plan.

\subsection{Route}
Route is named sequence of fixes. There can be some additional restrictions added on the route, for example altitude at given fixes or type of aircraft that can use the route. The flight plan does not need to include the whole route, the plane can be ordered to follow only part of the route between two fixes.

\subsubsection{STAR}
Standard instrument arrival (STAR). A designated instrument flight rule (IFR) arrival route linking a significant point,
normally on an ATS route, with a point from which a published instrument approach procedure can be commenced. \cite{doc4444}

\subsubsection{SID}
Standard instrument departure (SID). A designated instrument flight rule (IFR) departure route linking the aerodrome
or a specified runway of the aerodrome with a specified significant point, normally on a designated ATS route, at
which the en-route phase of a flight commences