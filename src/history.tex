%!TEX root = index.tex

\section{Brief History of Aviation}

The history of aviation began on 17th December 1903 with the first flight of Wright brother's powered fixed-wing airplane heavier than air. \cite{nolan} In first years after the maiden flight the aviation was considered only a dangerous pastime for daredevils, but year after year the early flying machines were becoming more capable and safe. By the end of World War I planes would prove themselves useful for observation and weapon delivery.

After war several uses were found for planes. One of them would be application of pesticides in agriculture, but the most important for further development of aviation would be probably delivery of mail. At this time most flights were conducted in daytime but with the rising demand for air mail delivery first experiments with flight in night were performed, first using bonfires for navigation and later replacing bonfires with gas and electrical lighting.

In 1930s commercial industry began to form in aviation and with increasing number of planes in air the need for some way of air traffic control became apparent. First new on-board instruments were designed to allow flight at certain altitude and direction without visual reference to the ground, and later system of ground radio navigational aids was introduced to assist pilots with navigation in low-visibility conditions. First air traffic controllers were located at airports where the traffic was most dense. They stood on well visible place on the airfield and would either permit the take-off or landing with a green flag or prohibit the pilot from proceeding with intended manoeuvre and hold position until it's safe to proceed.

This early control had many disadvantages and quickly evolved into system using light guns to guide the pilots near airfield. The basic principle was the same with green meaning ``go'' and red meaning ``stop'', only flags were replaced with focused light guns that allowed to better target the instructions to specific planes. Also the operation of light guns could be carried from elevated control tower which improved visibility for both pilots and controllers as well as provided more comfort for controllers.

Later two-way radio system between control tower and aircraft was introduced and allowed pilots to confirm issued instruction and controllers to transmit additional information regarding traffic, weather, etc. Unlike light guns that required direct line of sight between tower and aircraft, this system could be also used in low-visibility conditions.

Soon the airspace became more and more crowded and it was needed to control the traffic not only in the vicinity of airports but also on the routes between them. This was done by Air traffic control units (ATCUs) that would separate the air traffic on federal airways during instrument flying conditions. When the visibility was good it was still responsibility of pilots to separate their flight from surrounding traffic. The ATCUs were equipped with radio and airspace maps and would update the position of each aircraft on the map based on flight plan filled by the pilot before the flight and periodical position reports over the radio. This way the controllers at ATCU were able to keep the aircraft separated.

The World War II brought great advance in both aviation technology as well as the amount of air traffic. Approach control service was formed at the busiest airports, its task was to sequence the incoming and outcoming traffic so it will arrive to the airport in regular intervals and therefore making the task of tower control easier. The radar proved itself useful in military environment but needed some development to work well for civilian air traffic control. So in 1950s first air route surveillance radar became operational and the radar system was soon after improved with transponders on board of controlled aircraft that would allow to display the flight id and altitude on the radar screen.

New positioning system for aircraft was also developed around VHF omnidirectional range (VOR) and Distance measuring equipment (DME) using ground-based radio beacons. There were around 3000 VOR stations built around the world.\cite{vor} Because this system needs ground stations to operate it can't provide navigation over oceans or other places the stations cannot be built on.

The following years brought continuous evolution of the used equipment and procedures. Instrument landing system (ILS) was introduced to allow landing in adverse meteorological conditions. Computers soon found their way into control towers. They allowed the controllers to work with flight plans without printing them out and to transfer the FP between sectors automatically without using telephone which was used before. Computerized radar system was developed that would show additional information directly on the radar screen.

\section{Current ATM Equipment}

The current equipment of the traffic controllers is the result of the improvements made in the few previous decades. The computer system controllers use aggregates data from various sources and shows them conveniently on the screen. These data include aircraft positions computed from signals from multiple radar stations, flight id, altitude, speed and flight plan.

In order to be able to display the additional information the secondary radar system must be used.\cite{nolan} Standard radar emits radio waves and measures the interval between the pulse and when the waves reflected from any solid objects arrive back. This way the position of objects can be determined but the system is prone to interference and reflections from tall buildings, mountains or even cloud formations. The secondary radar also emits an interrogation signal and aircraft equipped with transponder will respond according to the interrogation mode. This way the plane's flight id and altitude can be shown on the radar screen. The aircraft speed is computed from a few previous positions of the aircraft and its altitude.

Another tool used in air traffic control nowadays is air traffic flow management (ATFM). This system predicts the air traffic density based on the available flight plans and if it reaches the capacity of the destination airport or sector the aircraft is delayed on the ground before it even takes of saving considerable amount of fuel. The process of the computing the airspace capacity utilization is very complex and influenced by many factors most important being the weather and is therefore automated and handled by computers.