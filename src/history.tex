%!TEX root = index.tex

\section{Brief History of Aviation}

The history of aviation began on 17th December 1903 with the first flight of Wright brother's powered fixed-wing airplane heavier than air. \cite{nolan} In first years after the maiden flight the aviation was considered only a dangerous pastime for daredevils, but year after year the early flying machines were becoming more capable and safe. By the end of World War I planes would prove themselves useful for observation and weapon delivery.

After war several uses were found for planes. One of them would be application of pesticides in agriculture, but the most important for further development of aviation would be probably delivery of mail. At this time most flights were conducted in daytime but with the rising demand for air mail delivery first experiments with flight in night were performed, first using bonfires for navigation and later replacing bonfires with gas and electrical lighting.

In 1930s commercial industry began to form in aviation and with increasing number of planes in air the need for some way of air traffic control became apparent. First new on-board instruments were designed to allow flight at certain altitude and direction without visual reference to the ground, and later system of ground radio navigational aids was introduced to assist pilots with navigation in low-visibility conditions. First air traffic controllers were located at airports where the traffic was most dense. They stood on well visible place on the airfield and would either permit the take-off or landing with a green flag or prohibit the pilot from proceeding with intended manoeuvre and hold position until it's safe to proceed.

This early control had many disadvantages and quickly evolved into system using light guns to guide the pilots near airfield. The basic principle was the same with green meaning ``go'' and red meaning ``stop'', only flags were replaced with focused light guns that allowed to better target the instructions to specific planes. Also the operation of light guns could be carried from elevated control tower which improved visibility for both pilots and controllers as well as provided more comfort for controllers.

Later two-way radio system between control tower and aircraft was introduced and allowed pilots to confirm issued instruction and controllers to transmit additional information regarding traffic, weather, etc. Unlike light guns that required direct line of sight between tower and aircraft, this system could be also used in low-visibility conditions.

Soon the airspace became more and more crowded and it was needed to control the traffic not only in the vicinity of airports but also on the routes between them. This was done by Air traffic control units (ATCUs) that would separate the air traffic on federal airways during instrument flying conditions. When the visibility was good it was still responsibility of pilots to separate their flight from surrounding traffic. The ATCUs were equipped with radio and airspace maps and would update the position of each aircraft on the map based on flight plan filled by the pilot before the flight and periodical position reports over the radio. This way the controllers at ATCU were able to keep the aircraft separated.

\red{TODO}

%  ve 40 letech zavedení enroute kontroly s FP na kartičkách, vylepšení radiomajáků, po 2.světové zavedení radaru, vylepšení radaru s odpovídačem který zobrazuje i id letadla, vylepšování a zavádění počítačů