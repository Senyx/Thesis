%!TEX root = index.tex
\chapter{Implementation}

\section{STAR Application}
\subsection{Flight Plan update}
\subsection{Clearances}
\subsection{Speed}

\section{TMA to Enroute Communication}

\section{Flow Control}
\subsection{Vectoring}
\subsection{Holding Pattern}
\subsection{Miles In Trails}

\section{Runway Plan Visualization}

% přivádím až k letišti ale poslední úsek jen tak nahrubo, pak ho odřídí tower

% přesnější popis zadání - so teda chci dosáhnout

% Class B airspace Around atlanta
% vytvořeno z dostupných informací - popsat tam slovně ten tvar?
% obrázky
% vnější obvod aby seděl na dané sektory
% informace o update tvaru

% první update:
% http://www.dot.ga.gov/localgovernment/intermodalprograms/aviation/documents/classbpresentationshowformat.pdf
% http://proofofright.files.wordpress.com/2011/07/atl-class-b1.jpg

% druhý update:
% http://web.co.dekalb.ga.us/pdkairport/pdf/AirspaceArticle.pdf
% asi aktuální verze:
% https://www.federalregister.gov/articles/2012/02/03/2012-2072/proposed-modification-of-the-atlanta-class-b-airspace-area-ga !!!!!!!!!!!!!!!!!!!!!

% nejdřív zkoušení funkčnosti na doplňku, pak na class 2 airspace, protože pod ním se nacházejí další letiště (12, v okolí dalších 11) -> menší prostor -> komplikovanější



% definice samotného letiště,
% v tuhle chvíli stačí runwaye
% podle
% http://airnav.com/airport/KATL a http://155.178.201.160/d-tpp/1410/00026AD.PDF

% dát tam samotný popis runwayí, obrázek konfigurace, co jsme vyignorovali (povrch, navádění atd.), screenshot z visia 





% vytvoření nové konfigurace, vyfiltrování letů, které letí z/do/přes atlantu

% popis výběru a napasování route na FP
% jak se řeší napasování routy když jde souběžně ale po jiných fixech, viz ASQ5502_798


% eventy, skenování, radio, přeplánování, update fpi a horizontálního plánu

% porovnání s online schedullingem, kde můžou tasky čekat dokud se procesor neuvolní, ale tady se musí vykonávat od začátku, protože jinak letadlo spadne

% generování volných slotů: co nejdřívější, bere se ohled na nepřesnost odhadu ETA a marginy na obou stranách, z nichž má přednost ten delší