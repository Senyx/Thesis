%!TEX root = index.tex

\documentclass[11pt,oneside,a4paper]{book}
%%%%\documentclass[11pt,twoside,a4paper]{book}   %%%%%%%%%%%%%%%%%%%%%%%%%%%%%%%%%%%%%%%%%%%%%%%%%%%%%%%%%%%%%%%%%%%%%%%%%%%%%%%%%


\usepackage[czech, english]{babel}
\usepackage[OT1]{fontenc} %[IL2] [T1] [OT1]
\usepackage[utf8]{inputenc}
\usepackage{lmodern}

\usepackage{graphicx}
\usepackage{color}


%\usepackage{indentfirst} %1. odstavec jako v cestine.

\usepackage{k336_thesis_macros}


\newcommand\TypeOfWork{Master's Thesis}   \typeout{Master's Thesis}
\newcommand\StudProgram{Open Informatics}
\newcommand\StudBranch{Software Engineering}


\newcommand\WorkTitle{Air Traffic Control Simulation in Terminal Area}
\newcommand\FirstandFamilyName{Bc. Jan Straka}
\newcommand\Supervisor{Mgr. Přemysl Volf, Ph.D}


\usepackage[
pdftitle={\WorkTitle},
pdfauthor={\FirstandFamilyName},
bookmarks=true,
colorlinks=true,
breaklinks=true,
urlcolor=red,
citecolor=blue,
linkcolor=blue,
unicode=true,
]
{hyperref}


\begin{document}
\selectlanguage{english} 


\typeout{************************************************}
\typeout{Language: english}
\typeout{Type of Work: \TypeOfWork}
\typeout{Study Program: \StudProgram}
\typeout{Study Branch: \StudBranch}
\typeout{Author: \FirstandFamilyName}
\typeout{Title: \WorkTitle}
\typeout{Supervisor: \Supervisor}
\typeout{***************************************************}

\newcommand\Department{Department of Computer Science and Engineering}
\newcommand\Faculty{Faculty of Electrical Engineering}
\newcommand\University{Czech Technical University in Prague}
\newcommand\labelSupervisor{Supervisor}
\newcommand\labelStudProgram{Study Programme} 
\newcommand\labelStudBranch{Field of Study}


% {
% \pagenumbering{roman} \cleardoublepage \thispagestyle{empty}
% \chapter*{Na tomto místě bude oficiální zadání vaší práce}
% \begin{itemize}
% \item Toto zadání je podepsané děkanem a vedoucím katedry,
% \item musíte si ho vyzvednout na studiijním oddělení Katedry počítačů na Karlově náměstí,
% \item v jedné odevzdané práci bude originál tohoto zadání (originál zůstává po obhajobě na katedře),
% \item ve druhé bude na stejném místě neověřená kopie tohoto dokumentu (tato se vám vrátí po obhajobě).
% \end{itemize}
% \newpage
% }


% \coverpagestarts

% \acknowledgements
% \noindent
% Zde můžete napsat své poděkování, pokud chcete a máte komu děkovat.


% \declaration{In Kořenovice nad Bečvárkou on May 15, 2008}

 
% \abstractpage
% Translation of Czech abstract into English.
% \vglue60mm
% \noindent{\Huge \textbf{Abstrakt}}
% \vskip 2.75\baselineskip
% \noindent
% Abstrakt práce by měl velmi stručně vystihovat její podstatu. Tedy čím se práce zabývá a co je jejím výsledkem/přínosem.
% \noindent
% Očekávají se cca 1 -- 2 odstavce, maximálně půl stránky.


% \tableofcontents
% \listoffigures
% \listoftables


\mainbodystarts
\normalfont
\parskip=0.2\baselineskip plus 0.2\baselineskip minus 0.1\baselineskip